
As said in this chapter introduction, the one presented is just a limited collection of possible experiments that, nevertheless, is enough to achieve a proper calibration and a full characterization.

In the following list, some other experiments and calibration routines will be briefly presented so that the reader/experimenter, eventually, can dive in on its own.


\begin{description}
    \item[Smearing correction:] this technique can improve the assignment fidelity~\cite{Krantz2016}: it consist in removing the initial and final part of the acquired waveform. Looking at \cref{fig:time_of_flight_2}, for example, we can see that in the transients (start and end of the pulse) the separation from signal to no-signal is not maximized. Through smearing we can remove those parts from the computation, achieving higher readout fidelities. 
    \item[Integration weights: ] again a routine for readout optimization~\cite{integration}. We usually acquire a certain waveform and then average all the points to a single one (in the IQ plane), this process is referred to as \textit{integration} and we are implicitly considering all the points as having the same importance. But we execute a time of flight experiment with state 0 and state 1, not every point will be separated the same. In particular, some points will be more significant and some less. By introducing weights for the integration of the waveform, we can exploit this to increase assignment fidelities (so the separation between the two states).
    \item[Direct calibration of \pihpulses:] with the Rabi and Flipping experiments, we calibrated the \pipulses to the best of our capabilities, however in several routines/experiments/circuits we end up using different pulses for different rotations. It may be worth to spend some time calibrating directly \pihpulses or even $\pi$/4-pulses to achieve finer tuned control and in the end higher gate fidelities. This could be done using experiments similar to the one already encountered as flipping, but only using sets of \pihpulses.
    \item[Readout-qubit frequency:] for flux tunable qubits we have found the sweetspot and we always used that as the bias level to operate them. In particular, this is required while driving the qubit since at different biases the frequency of the qubit is much more subject to noise changes. However, for reading out the qubit, so to interact with the resonator, it could be useful to use a different bias level, playing with the effective coupling qubit-resonator. This could potentially improve coherence times and fidelities.
    \item[Fast reset:] in the experiments we saw, after each singular shot we need to wait for the qubit to relax. Generally, this time is chosen to be around 5 times $T_1$. While working with qubits with low $T_1$ this is sustainable, but if $T_1$ is large, this can become a huge waste of time. It may be worth to implement and calibrate a \textit{fast reset} (or active reset) technique~\cite{Magnard2018}. The idea is that, instead of waiting, we measure the final state and check if it is zero or not\footnote{This must be done in real time, at the FPGA logic level, so must be supported by the device.}. If the qubit is already in $\ket 0$ we can run another shot of our experiment, otherwise we can fire a \pipulse and go on with another shot. Note that while this idea is extremely powerful, it may need a very fine calibration since every \pipulse has the possibility of mistakes. A possible way of reducing this problem is to measure multiple times, even after the \pipulse.
    \item[Memory-based reset:] a different approach to the same problem of fast reset is based on a memory approach. Instead of trying to reset the qubit to $\ket 0$, we can just record the measurement value and proceed to the next shot. If the measured value was $\ket 0$ nothing has to be done, otherwise the final result of the successive shot will require the application, in post-processing, of a X gate.
    \item[Quantum non-demolitioness of the readout: ] this parameter estimation is almost required for the active reset methods. Basically we always assumed that the readout pulses, since they are interacting only with resonators and not with qubits, are not changing its state. This is a fair assumption, but with active reset we require measurement that have QND probabilities $\gg 0.9$ or they will not make any sense. Therefore, it is important to compute the QND probability for a measurement~\cite{Gao2021}.
    \item[Cross-talk measurements:] when working with multiple qubits it is extremely interesting to measure the cross-talk matrix so, basically, how every qubit disturb drive and readout of the others. The field of cross-talk measurement has is roots in the field of noise modeling for qubits that is extremely vast and interesting.
    Different protocols~\cite{Barrett2023} already exist for this kind of experiment.
    \item[Quantum Zeno:] the Quantum Zeno effect~\cite{Harrington2017, Itano2009} basically allows to slow down the time evolution of a quantum system by the applying multiple frequent measurements. This can potentially lead to an increase of the effective $T_1$ and $T_2$, thus leading to better gate fidelities. Also this technique needs the measurements to be "fully" non-destructive.
    \item[Quantum tomography:] also called "full state tomography"~\cite{Altepeter2004} is the process to reconstruct completely a quantum state, by using multiple gates and measurements on different replica of the same qubit state. It is a powerful, and slow, tool to fully understand if a gate is behaving as expected.
\end{description}