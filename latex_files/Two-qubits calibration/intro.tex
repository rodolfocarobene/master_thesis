
After the last chapter, and with some more work and fine tuning, the experimenter should be able to reach a proper single qubit calibration.
However, to execute "useful" circuits, the capability of entangling multiple qubits and executing gates between them is critical.
Indeed, the real difference between classical and quantum computations \textit{is} the entanglement effect: from a probability point of view, all the results that we obtained are still compatible with classical theories (or hidden-variables theories) and we didn't yet encounter any real proof of quantum probability.

In this section, we will learn the principles of two-qubits gates and how to perform a basic calibration of those.
%The calibration presented here will be used in XXX to perform a Bell experiment and proof that this is indeed a quantum world.

There are two main two-qubits gates that can be implemented: CZ and iSWAP.
The calibration is more or less the same and the complexity to achieve them does not differ. 
I will focus on the iSWAP gate since I found better literature about it, but everything can be easily transposed for the CZ~\cite{Krantz2016, DiCarlo2009, McKay2017}.


The interaction of two qubits is done via a coupling capacity that produces an interaction term of:
\begin{equation}
    H_{qq} = g \sigma_{y1} \otimes \sigma_{y2}
\end{equation}
This Hamiltonian can be decomposed in $\sigma^\pm$ (ladder operators) and, dropping the fast rotating terms, we can arrive at:
\begin{equation}
    H_{qq} = g (e^{\delta \omega_{12}t}\sigma^+ \sigma^- + e^{-\delta \omega_{12}t}\sigma^- \sigma^+)
\end{equation}
where we have $\delta \omega_{12}= \omega_{q1} - \omega_{q2}$.

The last equation is not particularly clear and easy to understand, but this all changes when the two qubits share the same frequency:
\begin{equation}
    H_{qq}=g(\sigma^+\sigma^- + \sigma^-\sigma^+) = \frac{g}{2}(\sigma_x \sigma_x + \sigma_y \sigma_y)
\end{equation}
And this equation, when applied as a time evolution, produces an iSWAP gate.

The iSWAP gate can be written in the canonical basis as:
\begin{equation}
    iSWAP = 
    \begin{pmatrix}
        1 & 0 & 0 & 0 \\
        0 & 0 & -i & 0 \\
        0 & -i & 0 & 0 \\
        0 & 0 & 0 & 1
    \end{pmatrix}
\end{equation}
while the time evolution of $H_{qq}$ is:
\begin{equation}\label{eq:iswap_oscillations}
    U_{qq}(t) = e^{-i\frac{g}{2}(\sigma_x \sigma_x + \sigma_y \sigma_y)t}= 
    \begin{pmatrix}
        1 & 0 & 0 & 0 \\
        0 & \cos(gt) & -i\sin(gt) & 0 \\
        0 & -i\sin(gt) & \cos(gt) & 0 \\
        0 & 0 & 0 & 1
    \end{pmatrix}
\end{equation}
with exact equality for $T=\frac{\pi}{2g}$.

So, what we have just demonstrated can be summarized in saying that the iSWAP can be implemented simply by activating the two-qubits interaction for a fixed time.

